\documentclass[a4paper,12pt]{report}
\usepackage[francais]{babel}
\usepackage[utf8]{inputenc}
\usepackage[T1]{fontenc}
\usepackage{lmodern}
\usepackage[svgnames]{xcolor}
\usepackage{float} % figure
\usepackage{eurosym} % euro character
\usepackage{gensymb} % \degree symbol
\usepackage{graphicx}

\title{ADSILLH CMS Project - Delivrou}
\author{ROUBY Pierre-Antoine \\ TABARIE David \\ GRIMARD Mathieu \\
  ALLOUL Dorian}

\begin{document}

\maketitle
\tableofcontents

\part{Cahier des charges}
\chapter{Delivrou}
\textit{Delivrou} à pour but d'être un site généraliste de vente de livre
avec livraison à domicile. L'entreprise est de taille modeste et est amenée
à passer par des tiers pour la livraison.

Dans le contexte actuel, \textit{Delivrou} doit faire face à une forte
concurrence de la part des géants du web, comme Amazon.

\chapter{Les besoins}
Plusieur critaire devrons être respectés pour le choix de la technologie du
site.
Il faut que les livres puissent disposer d'un nom, d'une description, d'un prix;
ainsi que de la possibilité d'avoir des réductions pour les clients réguliés.

Il faut pouvoir syncroniser le site de e-commerce avec un systeme de gestion
de stock et de facturation comme un ERP.

Il faut pouvoir personnaliser l'interface avec un thème pour
\textit{Delivrou}, et de plus, ell doit être simple et agréable.

\part{Backlog}
\chapter{Backlog}
\begin{itemize}
\item[\textcolor{red}{Obligatoire} -] Format de données compatible avec la vente
  de livres,
\item[\textcolor{red}{Fort} -] Simple à utiliser pour les utilisateurs,
\item[\textcolor{green}{Faible} -] Simple pour les administrateurs,
\item[\textcolor{orange}{Fort} -] Synchronisation avec un ERP,
\item[\textcolor{red}{Obligatoire} -] Synchronisation avec un système de
  gestion de stocks,
\item[\textcolor{orange}{Medium}] Personalisation du thème.
\end{itemize}

\part{Solution}
\chapter{Comparaison des différentes solutions CMS}
Il existe de nombreux CMS (Content Management Systems) ou SGC (Système
de Gestion de Contenu) en français, certains sont libres (ou open source)
et d'autre sont propriétaires, il n'ont pas tous les mêmes fonctionnalités
et il est très important de choisir le CMS qui correspond le plus aux
besoins du projet.

\textit{Delivrou} est un site de vente de livres, il faut donc que le
CMS permette de faire des sites e-commerces. Il faut aussi pouvoir
gérer les stocks de livres. Étant donné que \textit{Delivrou} et une
petite entreprise nous allons nous contentrer sur les CMS open source,
pour avoir si besoin l'aide de la communauté. Nous voulons aussi un
CMS qui a une installation et une prise en main facile et rapide, nous
avons en effet pas le temps d'apprendre à utiliser un CMS complexe
dans le temps impartis pour l'exercice. Le CMS devra aussi pouvoir
gérer plusieurs comptes de connexion.

\section{WordPress}
WordPress est le CMS open source le plus utilisé sur le marché. Grâce
à sa grande communauté il contient énormément d’extensions, de plugins
et de widgets (environ 18 000) et de nombreux templates gratuits sont
disponibles sur cette plateforme. Son installation est simple et
rapide et son utilisation est intuitive.  Son succès est aussi son
plus grand défaut car cela en fait une cible des Hackers, il faut donc
toujours être à jour, cependant certaines mises à jours ne sont
parfois pas compatibles avec les extensions ce qui conduit à une
administration supplémentaire et parfois lourde. Son large choix de
possibilités en fait un CMS complexe, lourd et parfois confus.  Il est
adapté pour des petites structures de sites internets dont le contenu
change régulièrement, il peut aussi être utilisé pour les grands
projets mais il nécessitera beaucoup d'administration.

\section{TYPO3}
TYPO3 est le second CMS open source le plus utilisé, il est
régulièrement mis à jour et est amélioré par une équipe compétente. Il
dispose d'un système de gestion de contenu pour les entreprise.  Grâce
à sa grande communauté les novice peuvent obtenir rapidement de l'aide
lorsqu'ils rencontre des difficultés.  Il nécessite plus de temps
d'apprentissage que d'autre CMS mais lorsque qu'il est bien maîtrisé
il intégré directement des outils utiles pour les entreprises, CRM,
ERP, gestion des droits étendus.  Réaliser l’installation, la
configuration, et la maintenance est un travail important il n'est
donc pas vraiment adapté pour les petits projets.


\section{Joomla!}
Joomla! est le second CMS le plus utilisé. Il correspond aux
utilisateurs expérimenté comme débutant malgrés que son application
soit un peu plus exigente que celle de wordpress. Il embarque de
nombreuses fonctionnalités ce qui est un avantage pour les nouveaux
utilisateurs qui ne sont pas obligés de chercher des extensions. Sa
documentation bien faite et sa communauté active sont aussi d'une
grande aide pour les novice.  Le logiciel est entièrement orienté
objet ce qui permet aux utilisateur de créer leur propre extensions
divisé en 3 catégorie (plugins, composant et module).  Le processus de
gestion des droit n'est pas complet il peut toutefois être amélioré
par des extensions mais ces dernière ne sont malheureusement pas
toujours gratuites.

\section{Drupal}
Drupal est un CMS totalement open source et libre. Il dispose d'une
installation de base très légère mais elle peut être étendue par le
biais d’extensions. Ce CMS offre une grande possibilité de
personnalisation mais il offre aussi toutefois des distributions
pré-configurées.  Il nécessite un savoir faire solide car si les
fonctionnalités de base ne suffisent pas il faut installer les modules
supplémentaires via FTP et la mise à jour des extensions est
compliquée par un manque de compatibilité ascendante.  Drupal est
particulièrement adapté pour les plateformes sociales ou le contenu
est généré par les utilisateurs, grâce à son système flexible et
modulaire.

\section{Contao}

Contao est une solution claire et facile d'utilisation, elle permet de
rendre le contenu accessible en XHTML et HTML5. Malgré sa plus petite
communauté il n'est pas a délaisser.  Il dispose d'une large gamme de
fonctions dans sa configuration de base et peut en plus être
facilement étendu. il dispose de nombreux templates intégrés qui
diffèrent selons votre secteur d'activité ou ce que vous voulez faire
de votre CMS.  Sa sécurité est presque irréprochable grâce a un
système de mise à jour vraiment très simple. De base plus adapté pour
les petite et moyenne entreprise, il peut aussi convenir aux grandes
entreprises grâce à l'ajout de certaines extensions même si il ne sera
jamais adapté aux solutions d'entreprises complexes.

\chapter{Solution choisie}

Nous avons choisi la solutions prestashop car nous avons décidé de choisir
une solution e-cormmerce puisque notre site est un site de vente de livre.
Entre prestashop et Magento nous avons préféré utilisé prestashop car il est
plus simple d'accès et que pour les délais que nous avions cette solution
nécessiterais moins de temps d'apprentissage et de mise en place.

\chapter{Mise en place}
\section{L'hébergement}
Nous avons décidé d’hébergé notre site car nous voulons pouvoir tous le
modifier a distance.Mr Pierre-Antoine Rouby dispose d'un serveur sur lequel
nous allons installer notre prestashop. Nous devons donc mettre en place
un serveur apache ainsi qu'un gestionnaire de bases de données tel que mysql ou mariaDB.

\section{L'installation de prestashop}

>>>>>>> edbe5965c70a035059e5eca936e98690eed12381
Pour installer wordpress nous allons utiliser ce tuto :
https://www.digitalocean.com/community/tutorials/how-to-install-wordpress-with-lamp-on-ubuntu-16-04
Pour le template nous avons choisi un template commercial sur le site
de wordpress : https://wordpress.org/themes/shop-and-commerce/

\section{La configuration}
Nous avons tous des comptes administrateur.
Nous allons géré les droits, les formulaires et le contenue. 
\end{document}
