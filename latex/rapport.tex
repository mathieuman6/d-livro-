\documentclass[a4paper,12pt]{report}
\usepackage[francais]{babel}
\usepackage[utf8]{inputenc}
\usepackage[T1]{fontenc}
\usepackage{lmodern}
\usepackage[svgnames]{xcolor}
\usepackage{float} % figure
\usepackage{eurosym} % euro character
\usepackage{gensymb} % \degree symbol
\usepackage{graphicx}

\title{ADSILLH CMS Project - Delivrou}
\author{ROUBY Pierre-Antoine \\ TABARIE David \\ GRIMARD Mathieu \\
  ALLOUL Dorian}

\begin{document}

\maketitle
\tableofcontents

\part{Cahier des charges}
\chapter{Delivrou}
\textit{Delivrou} à pour but d'être un site généraliste de vente de livre
avec livraison à domicile. L'entreprise est de taille modeste et est amenée
à passer par des tiers pour la livraison.

Dans le contexte actuel, \textit{Delivrou} doit faire face à une forte
concurrence de la part des géants du web, comme Amazon. Pour sela elle compte
sur la fidélisation des clients et le service clientelles.

\chapter{Les besoins}
\section{Le public cible}
Le site s'adresse au grand public, avec les clasique de la litérature mais
aussi les nouvelle sorti. Le site doit disposser d'un large fois de libre en
stock et doit donc être relier à un système de gestions de stock car le large
choix enpèche de stocker de grande quantité de livre. Il faut donc renouveller
les stocks régulièrement en fonction des besions.

\section{Structure des données}
L'enregistrement des livres doit respecter un format minimal de:
\begin{itemize}
\item Nom
\item Description (Avec un résumer du comptenue et des informations pratique
  comme la taille du livre par example)
\item Autheur(s)
\item Date de parution
\item Éditeur
\item Prix
\end{itemize}

Ces données permettront de faciliter la recherche des oeuvres par les utilisateur
ainsi que le tris par la plateforme.

\section{Organisation}
Comme dit plus haut, il faut que le CMS utilisé puisse être synchronisable avec
un système de gestion de stock.

\begin{itemize}
\item Si un livre est acheté sur la plateforme il faut qu'il soit immédiatement
  rétiré du stock pour éviter que quelqu'un d'autre ne l'achête alors que le
  stock est vide.
\item Il faut aussi que les commandes soit passées directement lorsque le stock
  passe sous un niveaux critique.
\end{itemize}
Mais il doit aussi être synchronisable avec un système de gestion de
facturation de type ERP afin de générer toutes les factures de l'entreprise sur
la même interface ERP. Il faudra donc un type de facture spécifique pour les
clients de la plateforme e-commerce.

\chapter{Conclusion}
Plusieur critaires devrons être respectés pour le choix de la technologie du
site.
Il faut que les livres puissent disposer d'un nom, d'une description, d'une
date de parution d'un auteur (ou groupe d'auteur), un éditeur et d'un prix.
De plus la plateforme doit pouvoir donner la possibilité d'avoir des
réductions pour les clients réguliés ainsi que des reductions ponctuelles sur
des produits cibles.

Il faut pouvoir synchroniser le site de e-commerce avec un système de gestion
de stock et de facturation comme un ERP.

Il faut pouvoir personnaliser l'interface avec un thème pour
\textit{Delivrou}, et de plus, elle doit être simple et agréable.

\part{Backlog}
\chapter{Backlog}
\begin{itemize}
\item[\textcolor{red}{Obligatoire} -] Format de données compatible avec la vente
  de livres,
\item[\textcolor{red}{Fort} -] Simple à utiliser pour les utilisateurs,
\item[\textcolor{green}{Faible} -] Simple pour les administrateurs,
\item[\textcolor{orange}{Fort} -] Synchronisation avec un ERP,
\item[\textcolor{red}{Obligatoire} -] Synchronisation avec un système de
  gestion de stocks,
\item[\textcolor{orange}{Fort} - ] Personnalisation du thème.
\item[\textcolor{orange}{Fort} - ] Un site ergonomique et accessible sur toutes les plateformes
\item[\textcolor{green}{Faible} - ] Un design épuré et simple d'accès
\item[\textcolor{orange}{Fort} - ] Une barre de recherche pour avoir accès aux contenus du site
\item[\textcolor{green}{Faible} - ] Un panier pour mettre les articles
\item[\textcolor{orange}{Obligatoire} - ] Un formulaire avec les
  données personnelles et les adresses de livraison
\item[\textcolor{red}{Obligatoire} - ] Un système de paiement
\item[\textcolor{green}{Faible} - ] Un système d'abonnement à des newsletters
\item[\textcolor{orange}{Fort} - ] Le site doit être accessible dans plusieurs langues
\item[\textcolor{green}{Faible} - ] Sur la page d'accueil on doit voir les nouveautés
\item[\textcolor{orange}{Fort} - ] Création d'un logo personnalisé
\item[\textcolor{orange}{Fort} - ] Accompagnement GEO
\item[\textcolor{red}{Obligatoire} - ] Le site doit être hébergé sur un serveur pour y avoir accès en ligne
\item[\textcolor{red}{Obligatoire} - ] Un nom de domaine
\item[\textcolor{red}{Obligatoire} - ] Une équipe de développeurs pour que le site soit accessible pour avril
\item[\textcolor{red}{Obligatoire} - ] Maintenance du serveur
\end{itemize}

\part{Solution}
\chapter{Comparaison des différentes solutions CMS}
Il existe de nombreux CMS (Content Management Systems) ou SGC (Système
de Gestion de Contenu) en français, certains sont libres (ou open source)
et d'autre sont propriétaires, il n'ont pas tous les mêmes fonctionnalités
et il est très important de choisir le CMS qui correspond le plus aux
besoins du projet.

\textit{Delivrou} est un site de vente de livres, il faut donc que le
CMS permette de faire des sites e-commerces. Il faut aussi pouvoir
gérer les stocks de livres. Étant donné que \textit{Delivrou} et une
petite entreprise nous allons nous contentrer sur les CMS open source,
pour avoir si besoin l'aide de la communauté. Nous voulons aussi un
CMS qui a une installation et une prise en main facile et rapide, nous
avons en effet pas le temps d'apprendre à utiliser un CMS complexe
dans le temps impartis pour l'exercice. Le CMS devra aussi pouvoir
gérer plusieurs comptes de connexion.

\section{WordPress}
WordPress est le CMS open source le plus utilisé sur le marché. Grâce
à sa grande communauté il contient énormément d’extensions, de plugins
et de widgets (environ 18 000) et de nombreux templates gratuits sont
disponibles sur cette plateforme. Son installation est simple et
rapide et son utilisation est intuitive.  Son succès est aussi son
plus grand défaut car cela en fait une cible des Hackers, il faut donc
toujours être à jour, cependant certaines mises à jours ne sont
parfois pas compatibles avec les extensions ce qui conduit à une
administration supplémentaire et parfois lourde. Son large choix de
possibilités en fait un CMS complexe, lourd et parfois confus.  Il est
adapté pour des petites structures de sites internets dont le contenu
change régulièrement, il peut aussi être utilisé pour les grands
projets mais il nécessitera beaucoup d'administration.

\section{TYPO3}
TYPO3 est le second CMS open source le plus utilisé, il est
régulièrement mis à jour et est amélioré par une équipe compétente. Il
dispose d'un système de gestion de contenu pour les entreprise.  Grâce
à sa grande communauté les novice peuvent obtenir rapidement de l'aide
lorsqu'ils rencontre des difficultés.  Il nécessite plus de temps
d'apprentissage que d'autre CMS mais lorsque qu'il est bien maîtrisé
il intégre directement des outils utiles pour les entreprises, CRM,
ERP, gestion des droits étendus.  Réaliser l’installation, la
configuration, et la maintenance est un travail important il n'est
donc pas vraiment adapté pour les petits projets.

\section{Joomla!}
Joomla! est le second CMS le plus utilisé. Il correspond aux
utilisateurs expérimentés comme débutant malgré que son application
soit un peu plus exigente que celle de wordpress. Il embarque de
nombreuses fonctionnalités ce qui est un avantage pour les nouveaux
utilisateurs qui ne sont pas obligés de chercher des extensions. Sa
documentation bien faite et sa communauté active sont aussi d'une
grande aide pour les novice. Le logiciel est entièrement orienté
objet ce qui permet aux utilisateur de créer leur propres extensions
divisées en 3 catégorie (plugins, composant et module).  Le processus de
gestion des droit n'est pas complet il peut toutefois être amélioré
par des extensions mais ces dernière ne sont malheureusement pas
toujours gratuites.

\section{Drupal}
Drupal est un CMS totalement open source et libre. Il dispose d'une
installation de base très légère mais elle peut être étendue par le
biais d’extensions. Ce CMS offre une grande possibilité de
personnalisation mais aussi des distributions pré-configurées.  Il
nécessite un savoir faire solide car si les fonctionnalités de base ne
suffisent pas il faut installer les modules supplémentaires via FTP et
la mise à jour des extensions est compliquée par un manque de
compatibilité ascendante.  Drupal est particulièrement adapté pour les
plateformes sociales ou le contenu est généré par les utilisateurs,
grâce à son système flexible et modulaire.

\section{Contao}
Contao est une solution claire et facile d'utilisation, elle permet de
rendre le contenu accessible en XHTML et HTML5. Malgré sa plus petite
communauté il n'est pas à délaisser.  Il dispose d'une large gamme de
fonctions dans sa configuration de base et peut en plus être
facilement étendu. Il dispose de nombreux templates intégrés qui
diffèrent selons le secteur d'activité ciblé. Sa sécurité est presque
irréprochable grâce a un système de mise à jour vraiment très
simple. De base plus adapté pour les petite et moyenne entreprise, il
peut aussi convenir aux grandes entreprises grâce à l'ajout de
certaines extensions même si il ne sera jamais adapté aux solutions
d'entreprises complexes.

\chapter{Solution choisie}
Nous avons choisi la solutions Prestashop car nous avons décidés de
nous tourner vers une solution e-commerce puisque notre site est un
site de vente de livre.  Entre prestashop et Magento nous avons
préférés utiliser Prestashop car il est plus simple d'accès et que
pour les délais que nous avions cette solution nécessiterais moins de
temps d'apprentissage et de mise en place. De plus il rempli toutes
les exigence du cahier des charges.

\chapter{Mise en place}
\section{L'hébergement}
Nous avons décidé d’héberger notre site car nous voulions pouvoir tous
le modifier à distance. Mr Pierre-Antoine Rouby dispose d'un serveur
sur lequel nous allons installer notre Prestashop. Nous devons donc
mettre en place un serveur apache ainsi qu'un gestionnaire de bases de
données tel que mysql ou mariaDB.

\section{L'installation de prestashop}


\section{La configuration}
Nous avons tous des comptes administrateur.
Nous allons gêrer les droits, les formulaires et le contenue.

\newpage
\section{Budget Prévisionnel}
Développement initial……………………………………….....2100\euro  \\
\hangindent=1.5cm coût de base.....................................................................................500\euro  \\
design…………………………………………...………500\euro  \\
fonctionnalités:...............................................................................1100\euro  \\

Frais Récurrents (/mois)…………………………....…………..1800\euro  \\
\hangindent=1.5cm Hébergement……………..........………………………….240\euro  \\
Maintenance……………...…...………………………….1560\euro  \\

Services Additionnels……………………….....……………….1000\euro  \\
\hangindent=1.5cm    Création logo….....……………………………………….200\euro  \\
    Accompagnement SEO……………………………...............800\euro  \\
\\\\\\
TOTAL………………………......................................……4900\euro  \\
\end{document}
